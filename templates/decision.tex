\documentclass[11pt,twoside]{protokoll2}
%\usepackage{bookman}
%\usepackage{newcent}
%\usepackage{palatino}
\usepackage{pdfpages}
\usepackage{eurosym}
%\usepackage[utf8]{inputenc}
\usepackage[pdfborder={0 0 0}]{hyperref}
%\usepackage{ngerman}
% \usepackage[left]{lineno}
%\usepackage{footnote}
%\usepackage{times}
\renewcommand{\thefootnote}{\fnsymbol{footnote}}
\renewcommand{\thempfootnote}{\fnsymbol{mpfootnote}}
%\renewcommand{\familydefault}{\sfdefault}
\newcommand{\einrueck}[1]{\hfill\begin{minipage}{0.95\linewidth}#1\end{minipage}}


\begin{document}
%\thispagestyle{plain}   %ggf kommentarzeichen entfernen
\Titel{
\large Protokoll: \VAR{protocol.protocoltype.name|escape_tex}
\\\normalsize \VAR{protocol.protocoltype.organization|escape_tex}
}{}
\begin{tabular}{rp{14cm}}
\ENV{if protocol.date is not none}
    {\bf Datum:} & \VAR{protocol.date|datify_long|escape_tex}\\
\ENV{endif}
\ENV{for meta in protocol.metas}
    {\bf \VAR{meta.name|escape_tex}:} & \VAR{meta.value|escape_tex}\\
\ENV{endfor}
\end{tabular}
\normalsize

\section*{Beschluss}
\begin{itemize}
    \item \VAR{decision.content|escape_tex}
\end{itemize}

\setcounter{section}{\VAR{top.get_top_number() - 1}}
\TOP{\VAR{top.name|escape_tex}}
\VAR{top.render(render_type=render_type, level=0, show_private=show_private, protocol=protocol)}

\end{document}
