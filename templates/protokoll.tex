\documentclass[11pt,twoside]{protokoll2}
%\usepackage{bookman}
%\usepackage{newcent}
%\usepackage{palatino}
\usepackage{pdfpages}
\usepackage{eurosym}
\usepackage[utf8]{inputenc}
\usepackage[pdfborder={0 0 0}]{hyperref}
\usepackage{ngerman}
% \usepackage[left]{lineno}
%\usepackage{footnote}
%\usepackage{times}
\renewcommand{\thefootnote}{\fnsymbol{footnote}}
\renewcommand{\thempfootnote}{\fnsymbol{mpfootnote}}
%\renewcommand{\familydefault}{\sfdefault}
\newcommand{\einrueck}[1]{\hfill\begin{minipage}{0.95\linewidth}#1\end{minipage}}

\begin{document}
%\thispagestyle{plain}   %ggf kommentarzeichen entfernen
\Titel{
\large Protokoll: \VAR{protocol.protocoltype.name|escape_tex}
\\\normalsize \VAR{protocol.protocoltype.organization|escape_tex}
}{}
\begin{tabular}{rp{15.5cm}}
{\bf Datum:} & \VAR{protocol.date|datify_long|escape_tex}\\
{\bf Ort:} & \VAR{protocol.location|escape_tex}\\
{\bf Protokollant:} & \VAR{protocol.author|escape_tex}\\
{\bf Anwesend:} & \VAR{protocol.participants|join(", ")|escape_tex}\\
\end{tabular}
\normalsize

\section*{Beschlüsse}
\begin{itemize}
\ENV{if protocol.decisions|length > 0}
    \ENV{for decision in protocol.decisions}
        \item \VAR{decisions.content|escape_tex}
    \ENV{endfor}
\ENV{else}
	\item Keine Beschlüsse
\ENV{endif}
\end{itemize}

Beginn der Sitzung: \VAR{protocol.start_time|timify}

\ENV{for top in tree.children}
    \TOP{\VAR{top.name}} % here we probably have information doubly
    \VAR{top.render()}
\ENV{endfor}

Ende der Sitzung: \VAR{protocol.end_time|timify}

\end{document}
